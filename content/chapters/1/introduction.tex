% !TEX root = ../../../main.tex
%

\chapter{Cablaggio Strutturato}

\section{Requisiti}
È richiesto un cablaggio standard ISO/IEC11801 con 2 prese in rame per ogni posto di lavoro.
In aggiunta alla topologia stellare è richiesta l’introduzione di collegamenti in rame (almeno 4 cavi da 4 coppie)
tra gli armadi adiacenti, per la realizzazione di reti fisiche di estensione limitata in piccole zone dell’edificio
e per eventuali cammini ridondanti per soluzioni fault tolerant.
Uno dei locali dell’edificio (adeguatamente indicato nelle planimetrie) dovrà essere adibito a sala macchine e
ospiterà i server.
Un centralino telefonico sarà ospitato nel vano al piano terreno che ospiterà anche l’armadio di centro stella di
edificio; in tale vano arriveranno i collegamenti ai servizi esterni.

\section{Planimetrie}
Si riportano di seguito le planimetrie dell'edificio opportunamente annotate e dimensionate.
Le quote sono approssimative in quanto non fornite nelle specifiche originali.

Nella planimetria si è usato uno spessore delle pareti pari a \SI{40}{\centi\metre} per quelle esterne e di \SI{15}{\centi\metre} per
quelle interne.